\documentclass[a4paper,11pt]{jsarticle}
\usepackage[a4paper]{geometry}
\usepackage{multirow}

\newcommand{\showVer}{Ver 1.0}
\newcommand{\midTrue}{○}
\newcommand{\midFalse}{×}

\begin{document}
\begin{titlepage}
\title{HTLAB.NET Arduino DRSSTC Interrupter \\\showVer}
\author{\vspace*{80mm}\ \\つくば科学株式会社\\菊地 秀人}
\date{\vspace*{20mm}\ \\2017年12月03日}
\maketitle
\thispagestyle{empty}
\end{titlepage}
\clearpage

\section{イントロダクション}
この文書では、「HTLAB.NET Arduino DRSSTC Interrupter」の使い方について解説します。\\
プログラムのビルド内容によって操作方法が変わるため、プログラム内容の解説から始めます。




\section{プログラムの基本構造}
\subsection{要求するハードウェア}
このプログラムは以下のボードで動作します。\\
搭載マイコンの種類によって、出力数の違い、USB-MIDI対応などが異なります。

\begin{table}[htbp]
\begin{center}
\begin{tabular}{ | c | c | c | c | }
\hline
\textbf{ボード名} & \textbf{搭載マイコン} & \textbf{出力数} & \textbf{USB-MIDI対応} \\\hline
\textbf{Arduino Uno R3} & ATmega328P & 1 & 外部ファームウェアMoco適用で可能\\\hline
\textbf{Arduino Leonardo} & ATmega32u4 & 2 & 対応 \\\hline
\textbf{Arduino Micro} & ATmega32u4 & 2 & 対応 \\\hline
\end{tabular}
\end{center}
\end{table}

\subsection{要求するソフトウェア}
Arduino IDE 1.8.1 以降
https://www.arduino.cc/en/Main/Software

Arduino MIDI Library 4.3.1 以降
https://playground.arduino.cc/Main/MIDILibrary

MIDIUSB library 1.0.3 以降
https://www.arduino.cc/en/Reference/MIDIUSB



\subsection{ファイル構成}

\begin{table}[htbp]
\begin{center}
\begin{tabular}{ | c | c | }
\hline
\textbf{ファイル名} & \textbf{ファイル内容} \\\hline
\textbf{HTLAB.NET\_Arduino\_DRSSTC\_Interrupter.ino} & Arduinoメインプログラム \\\hline
\textbf{lib\_midi.h} & MIDI関係定義ヘッダファイル \\\hline
\textbf{lib\_osc.cpp} & 発振器操作に関するプログラム \\\hline
\textbf{lib\_osc.h} & 発振器操作に関するヘッダファイル \\\hline
\textbf{lib\_output.cpp} & ピン出力に関するプログラム \\\hline
\textbf{lib\_output.h} & ピン出力に関するヘッダファイル \\\hline
\textbf{settings.h} & 設定定義ヘッダファイル \\\hline
\end{tabular}
\end{center}
\end{table}





\clearpage
\newgeometry{top=1truecm, bottom=1truecm, left=1truecm, right=1truecm}

\begin{center}
\huge{MIDIインプリメンテーションチャート}
\end{center}

\begin{flushright}
\small{HTLAB.NET Arduino DRSSTC Interrupter \showVer}
\end{flushright}

\begin{table}[htbp]
\begin{center}
\begin{tabular}{ | p{2cm} p{2cm} | p{4cm} | p{4cm} | p{4cm} | }
\hline
\multicolumn{2}{|c|}{\textbf{\large ファンクション}} & 
\multicolumn{1}{c|}{\textbf{\large 送信}} & 
\multicolumn{1}{c|}{\textbf{\large 受信}} & 
\multicolumn{1}{c|}{\textbf{\large 備考}} \\\hline


\textbf{ベーシック} & \multicolumn{1}{r|}{電源ON時} & \midFalse & 1-2 & チャンネル変更不可 \\
\textbf{チャンネル} & \multicolumn{1}{r|}{設定可能} & \midFalse & \midFalse & \\\hline

 & \multicolumn{1}{r|}{電源ON時} & \midFalse & モード3 & \\
\textbf{モード} & \multicolumn{1}{r|}{メッセージ} & \midFalse & \midFalse & \\
 & \multicolumn{1}{r|}{代用} & \midFalse & \midFalse & \\\hline

\textbf{ノート} & & \midFalse & 0-127 & 制限可能 \\
\textbf{ナンバー} & \multicolumn{1}{r|}{:音域} & \midFalse & 0-127 & タイマ8分周時下限24 \\\hline

\multirow{2}{*}{\textbf{ベロシティ}} & \multicolumn{1}{r|}{ノート・オン} & \midFalse & 0-127 & 使用設定による \\
 & \multicolumn{1}{r|}{ノート・オフ} & \midFalse & \midFalse &  \\\hline

\textbf{アフター} & \multicolumn{1}{r|}{キー別} & \midFalse & \midFalse & \\
\textbf{タッチ} & \multicolumn{1}{r|}{チャンネル別} & \midFalse & \midFalse & \\\hline

\textbf{ピッチベンド} & & \midFalse & \midFalse & \\\hline




\end{tabular}
\end{center}
\end{table}
\thispagestyle{empty}
\restoregeometry




\end{document}


